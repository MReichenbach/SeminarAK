\section*{Fazit}~\label{Kapitel Fazit}
	TODO Fazit....
	
	\wup Bezüglich der Sicherheit von heutigen kryptographischen Verfahren muss man sich aktuell keine Sorgen machen, zumindest wenn die Mindestanforderungen bezüglich der Größe des Körpers und der Schlüssellänge eingehalten werden. Bei zu kleinen Schlüssellängen ist auch bei der besten Verschlüsselung heutzutage keine Sicherheit gegeben.
	
	\wupp Der Höhepunkt im Bereich der Verschlüsselung kann noch nicht erreicht sein. Noch schnellere und effizientere und somit \myAnfuehrungszeichen{sicherere} Verschlüsselungstechniken werden benötigt um auch zukünftigen Anforderungen gerecht zu werden. Denn die nötigen Algorithmen um sämtliche heutigen Verschlüsselungen zu brechen existieren schon, laut Peter W. Shor. Diese stellt er in \cite{Algorithms:for:Quantum:Computation:Discrete:Logarithms:and:Factoring} vor. Diskrete Logarithmen können in polynomialer Zeit berechnet werden, es muss nur noch gelingen einen Quantencomputer zu bauen.