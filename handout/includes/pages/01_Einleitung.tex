\chapter*{Algorithmische Zahlentheorie}\label{Algorithmische Zahlentheorie}
	Diese Arbeit beschäftigt sich mit der theoretischen Grundlage für die asymmetrische Kryptographie. Es werden zunächst die benötigten Grundlagen der Zahlentheorie gegeben, um anschließend einen genaueren Blick auf die Algorithmen zur Primzahlerkennung, des Diffie-Hellmann-Austausches und der Anwendung elliptischer Kurven zu geben. Neben der Ausarbeitung verwendeter Algorithmen wird auch die Laufzeit dieser untersucht. Zufallszahlengeneratoren werden nicht betrachtet.
		
	\wupp Im folgenden werden die Grundlagen der Zahlentheorie eingeführt, um in der Präsentation ausgewählte Algorithmen, die in der asymmetrischen Kryptographie eingesetzt werden, betrachten zu können. Zum Nachschlagen weiterer Informationen über Persönlichkeiten der Zahlentheorie mit historischer Einordnung bietet das Buch \cite{Elementare:Zahlentheorie} von Jochen Ziegenbalg einen guten Überblick.