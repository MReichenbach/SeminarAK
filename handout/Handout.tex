\documentclass[deutsch, fontsize=11pt,a4paper,onecolumn]{scrreprt}

%Using-Statements

\usepackage[usenames, dvipsnames]{xcolor}
\usepackage[utf8]{inputenc}
\usepackage [ ngerman ] { babel }
\usepackage[pdfborder={0 0 0}] {hyperref}
\usepackage{graphicx}
\usepackage{booktabs}
\usepackage[style=numeric, backend=bibtex,maxnames=10]{biblatex}
\usepackage[babel]{csquotes}
\usepackage{listings}
\usepackage{varioref}
\usepackage{epsfig}
\usepackage{float}
\usepackage{prettyref}
\usepackage{titleref}
\usepackage{xspace}
\usepackage{datatool}
\usepackage{color}
\usepackage{todonotes}
\usepackage[grey, palatino]{quotchap}
\usepackage{rotating}
\usepackage[automark]{scrpage2}
\usepackage{shadethm}
\usepackage{svg}
\usepackage{scrpage2} \pagestyle{scrheadings}
\clearscrheadfoot \ohead{\pagemark} \ihead{Beispiel}
\setheadsepline{0.4pt} 
\ihead{\headmark}
\usepackage[subject={Top1},author={\AA{}nsgar Lund},version=1]{pdfcomment}

\usepackage{amssymb}
\usepackage{stmaryrd}

\usepackage{titlesec}

\newcommand*{\justifyheading}{\raggedright}
\titleformat{\chapter}[display]
{\normalfont\huge\bfseries\justifyheading}{\chaptertitlename\ \thechapter}{20pt}{}

%Settings
%%% Refs %%%
\newrefformat{sec}{Section~\ref{#1} \glqq\titleref{#1}\grqq \ on page \pageref{#1}}
\bibliography{sources}
\vrefwarning
\setcounter{tocdepth}{2} 
\setcounter{secnumdepth}{2}
%\frenchspacing                  % no extra space after full stops

%Definitions
\newcommand{\HRule}{\rule{\linewidth}{1.0mm}}
\newcommand{\includecode}[2][c]{\lstinputlisting[caption=#2, escapechar=, style=#1]{#2}}
\newcommand{\wupp}{\bigskip \noindent}
\newcommand{\wup}{\noindent}
\newcommand{\bolditem}[1]{\item \textbf{#1}}
\newcommand{\striche}[1]{\glqq{}#1\grqq{}}
\renewcommand\chapterheadstartvskip{\vspace*{-3\baselineskip}}
\renewcommand*{\chapterpagestyle}{chapter} 
\newcommand*{\refFull}[1]{\hyperref[{#1}]{\textbf{\ref*{#1} \nameref*{#1}}}} % One single link
\newcommand*{\refName}[1]{\hyperref[{#1}]{\textbf{\nameref*{#1}}}} % One single link
\newcommand{\pic}[5][\empty]{
\begin{figure}[h]
	#1
	\begin{minipage}[b]{0.5\textwidth} 
   		\includegraphics[#2]{#3}
   		\caption{#4}  
    	#5
    \end{minipage}
    
\end{figure}}

\newcommand{\mycirc}{$\circ~$}
\newcommand{\mycircOhne}{$\circ$}
\newcommand{\mycdot}{$\cdot~$}
\newcommand{\mycdotOhne}{$\cdot$}
\newcommand{\myin}{$\in~$}
\newcommand{\myMenge}[1]{$\mathbb{#1}$}
\newcommand{\myInfty}{$\infty~$}
\newcommand{\myInftyOhne}{$\infty$}
\newcommand{\myTiefstellen}[1]{$\mathrm{_#1}$}
\newcommand{\myHochstellen}[1]{$\mathrm{^#1}$}
\newcommand{\myTeiler}{$\mathrm{\mid}$~}
\newcommand{\myNichtTeiler}{$\mathrm{\nmid}$~}
\newcommand{\myKleinerGleich}{$\mathrm{\leq}$~}
\newcommand{\myGroesser}{$\mathrm{>}$~}
\newcommand{\myGroesserGleich}{$\mathrm{\geq}$~}
\newcommand{\myPhi}{$\mathrm{\varphi}$}
\newcommand{\myZPStern}{\myMenge{Z}\myTiefstellen{p^*}}
\newcommand{\myMathRM}[1]{$\mathrm{#1}$}
\newcommand{\myRefGleichung}[1]{(\ref{#1})}
\newcommand{\myAnfuehrungszeichen}[1]{\glqq{}#1\grqq{}}

\definecolor{mygreen}{rgb}{0,0.6,0}
\definecolor{mygray}{rgb}{0.5,0.5,0.5}
\definecolor{mymauve}{rgb}{0.58,0,0.82}
\definecolor{myHellGelb}{RGB}{255,255,200}

\lstdefinestyle{customxml}{
   belowcaptionskip=1\baselineskip,
   breaklines=true,
   frame=L,
   xleftmargin=\parindent,
   language=XML,
   showstringspaces=false,
   basicstyle=\footnotesize\ttfamily,
      keywordstyle=\bfseries\color{green!40!black},
      stringstyle=\color{blue},
       morestring=[b]",
  morestring=[s]{>}{<},
   	literate=
   	*{:}{{{\color{orange}{:}}}}{1}
   	{name}{{{\color{orange}{name}}}}{4}
   	{\{}{{{\color{delim}{\{}}}}{1}
   	{\}}{{{\color{delim}{\}}}}}{1}
   	{[}{{{\color{delim}{[}}}}{1}
   	{]}{{{\color{delim}{]}}}}{1}
}

\lstdefinestyle{JavaScript}{
	belowcaptionskip=1\baselineskip,
	xleftmargin=\parindent,
   	breaklines=true,
   	frame=L,
    basicstyle=\footnotesize\ttfamily,
   keywordstyle=\bfseries\color{green!40!black},
   commentstyle=\itshape\color{purple!40!black},
   identifierstyle=\color{blue},
   stringstyle=\color{black},
  	morestring=[d]',
  	morestring=[b]",
  	showstringspaces=false,
 	literate=
     *{:}{{{\color{punct}{:}}}}{1}
      {,}{{{\color{punct}{,}}}}{1}
      {\{}{{{\color{delim}{\{}}}}{1}
      {\}}{{{\color{delim}{\}}}}}{1}
      {[}{{{\color{delim}{[}}}}{1}
      {]}{{{\color{delim}{]}}}}{1}
}

\lstdefinestyle{XMLSchemaInText}{
	backgroundcolor=\color{white},   % choose the background color; you must add \usepackage{color} or \usepackage{xcolor}
	basicstyle=\footnotesize\ttfamily,        % the size of the fonts that are used for the code
	breakatwhitespace=false,         % sets if automatic breaks should only happen at whitespace
	breaklines=true,                 % sets automatic line breaking
	captionpos=b, 					% sets the caption-position to bottom                 
	commentstyle=\color{mygreen},    % comment style
	deletekeywords={},            % if you want to delete keywords from the given language
	escapeinside={\%*}{*)},          % if you want to add LaTeX within your code
	extendedchars=true,              % lets you use non-ASCII characters; for 8-bits encodings only, does not work with UTF-8
	frame=none,                    % adds a frame around the code
	keepspaces=true,                 % keeps spaces in text, useful for keeping indentation of code (possibly needs columns=flexible)
	keywordstyle=\color{black},       % keyword style
	language=XSLT,                 % the language of the code
	morekeywords={xsd:element, xsd:complexType, xsd:sequence, xsd:attributeGroup},            % if you want to add more keywords to the set
	numbers=none,                    % where to put the line-numbers; possible values are (none, left, right)
	numbersep=7pt,                    % how far the line-numbers are from the code
	numberstyle=\tiny\color{mygray}, % the style that is used for the line-numbers
	rulecolor=\color{black},         % if not set, the frame-color may be changed on line-breaks within not-black text (e.g. comments (green here))
	showspaces=false,                % show spaces everywhere adding particular underscores; it overrides 'showstringspaces'
	showstringspaces=false,          % underline spaces within strings only
	showtabs=false,                  % show tabs within strings adding particular underscores
	stepnumber=1,                    % the step between two line-numbers. If it's 1, each line will be numbered
	stringstyle=\color{black},     % string literal style
	identifierstyle=\color{black},
	tabsize=2,                       % sets default tabsize to 2 spaces
	title=\lstname                   % show the filename of files included with \lstinputlisting; also try caption instead of title
}

\lstdefinestyle{XMLSchemaInAnhang}{
	backgroundcolor=\color{myHellGelb},   % choose the background color; you must add \usepackage{color} or \usepackage{xcolor}
	basicstyle=\footnotesize\ttfamily,        % the size of the fonts that are used for the code
	breakatwhitespace=false,         % sets if automatic breaks should only happen at whitespace
	breaklines=true,                 % sets automatic line breaking
	captionpos=b, 					% sets the caption-position to bottom                 
	commentstyle=\color{mygreen},    % comment style
	deletekeywords={},            % if you want to delete keywords from the given language
	escapeinside={\%*}{*)},          % if you want to add LaTeX within your code
	extendedchars=true,              % lets you use non-ASCII characters; for 8-bits encodings only, does not work with UTF-8
	frame=single,                    % adds a frame around the code
	keepspaces=true,                 % keeps spaces in text, useful for keeping indentation of code (possibly needs columns=flexible)
	keywordstyle=\color{blue},       % keyword style
	language=XSLT,                 % the language of the code
	morekeywords={xsd:element, xsd:complexType, xsd:sequence, xsd:attributeGroup, xsd:complexContent, xsd:extension},            % if you want to add more keywords to the set
	numbers=left,                    % where to put the line-numbers; possible values are (none, left, right)
	numbersep=7pt,                    % how far the line-numbers are from the code
	numberstyle=\tiny\color{mygray}, % the style that is used for the line-numbers
	rulecolor=\color{mygray},         % if not set, the frame-color may be changed on line-breaks within not-black text (e.g. comments (green here))
	showspaces=false,                % show spaces everywhere adding particular underscores; it overrides 'showstringspaces'
	showstringspaces=false,          % underline spaces within strings only
	showtabs=false,                  % show tabs within strings adding particular underscores
	stepnumber=1,                    % the step between two line-numbers. If it's 1, each line will be numbered
	stringstyle=\color{black},     % string literal style
	identifierstyle=\color{blue},
	literate=
	{/}{{{\color{mygray}{/}}}}{1}
	{<}{{{\color{mygray}{<}}}}{1}
	{>}{{{\color{mygray}{>}}}}{1}
	{=}{{{\color{mygray}{=}}}}{1}
	{name}{{{\color{cyan}{name}}}}{4}
	{namespace}{{{\color{cyan}{namespace}}}}{8}
	{type}{{{\color{cyan}{type}}}}{4}
	{maxOccurs}{{{\color{cyan}{maxOccurs}}}}{8}
	{minOccurs}{{{\color{cyan}{minOccurs}}}}{8}
	{base}{{{\color{cyan}{base}}}}{4}
	{base-id}{{{\color{black}{base-id}}}}{6}
	{-base-}{{{\color{black}{-base-}}}}{5}
	{"type"}{{{\color{black}{"type"}}}}{5}
	{//}{{{\color{black}{//}}}}{2}
	{/2}{{{\color{black}{/2}}}}{2}
	{/1}{{{\color{black}{/1}}}}{2}
	{1/}{{{\color{black}{1/}}}}{2}
	{2/}{{{\color{black}{2/}}}}{2}
	{ref}{{{\color{cyan}{ref}}}}{3},
	tabsize=2                       % sets default tabsize to 2 spaces
%	title=\caption                   % show the filename of files included with \lstinputlisting; also try caption instead of title
}

\lstdefinestyle{XMLSchemaInText2}{
	backgroundcolor=\color{White},   % choose the background color; you must add \usepackage{color} or \usepackage{xcolor}
	basicstyle=\footnotesize\ttfamily,        % the size of the fonts that are used for the code
	breakatwhitespace=false,         % sets if automatic breaks should only happen at whitespace
	breaklines=true,                 % sets automatic line breaking
	captionpos=b, 					% sets the caption-position to bottom                 
	commentstyle=\color{mygreen},    % comment style
	deletekeywords={},            % if you want to delete keywords from the given language
	escapeinside={\%*}{*)},          % if you want to add LaTeX within your code
	extendedchars=true,              % lets you use non-ASCII characters; for 8-bits encodings only, does not work with UTF-8
	frame=none,                    % adds a frame around the code
	keepspaces=true,                 % keeps spaces in text, useful for keeping indentation of code (possibly needs columns=flexible)
	keywordstyle=\color{blue},       % keyword style
	language=XSLT,                 % the language of the code
	morekeywords={xsd:element, xsd:complexType, xsd:sequence, xsd:attributeGroup, xsd:complexContent, xsd:extension},            % if you want to add more keywords to the set
	numbers=none,                    % where to put the line-numbers; possible values are (none, left, right)
	numbersep=7pt,                    % how far the line-numbers are from the code
	numberstyle=\tiny\color{mygray}, % the style that is used for the line-numbers
	rulecolor=\color{mygray},         % if not set, the frame-color may be changed on line-breaks within not-black text (e.g. comments (green here))
	showspaces=false,                % show spaces everywhere adding particular underscores; it overrides 'showstringspaces'
	showstringspaces=false,          % underline spaces within strings only
	showtabs=false,                  % show tabs within strings adding particular underscores
	stepnumber=1,                    % the step between two line-numbers. If it's 1, each line will be numbered
	stringstyle=\color{black},     % string literal style
	identifierstyle=\color{blue},
	literate=
	{/}{{{\color{mygray}{/}}}}{1}
	{<}{{{\color{mygray}{<}}}}{1}
	{>}{{{\color{mygray}{>}}}}{1}
	{=}{{{\color{mygray}{=}}}}{1}
	{name}{{{\color{cyan}{name}}}}{4}
	{namespace}{{{\color{cyan}{namespace}}}}{8}
	{type}{{{\color{cyan}{type}}}}{4}
	{maxOccurs}{{{\color{cyan}{maxOccurs}}}}{8}
	{minOccurs}{{{\color{cyan}{minOccurs}}}}{8}
	{base}{{{\color{cyan}{base}}}}{4}
	{base-id}{{{\color{black}{base-id}}}}{6}
	{-base-}{{{\color{black}{-base-}}}}{5}
	{"type"}{{{\color{black}{"type"}}}}{5}
	{//}{{{\color{black}{//}}}}{2}
	{/2}{{{\color{black}{/2}}}}{2}
	{/1}{{{\color{black}{/1}}}}{2}
	{1/}{{{\color{black}{1/}}}}{2}
	{2/}{{{\color{black}{2/}}}}{2}
	{ref}{{{\color{cyan}{ref}}}}{3},
	tabsize=2                       % sets default tabsize to 2 spaces
	%	title=\caption                   % show the filename of files included with \lstinputlisting; also try caption instead of title
}

\lstdefinestyle{prologInText}{
	backgroundcolor=\color{white},   % choose the background color; you must add \usepackage{color} or \usepackage{xcolor}
	basicstyle=\footnotesize\ttfamily,        % the size of the fonts that are used for the code
	breakatwhitespace=false,         % sets if automatic breaks should only happen at whitespace
	breaklines=true,                 % sets automatic line breaking
	captionpos=b,                    % sets the caption-position to bottom
	commentstyle=\color{mygreen},    % comment style
	deletekeywords={},            % if you want to delete keywords from the given language
	escapeinside={\%*}{*)},          % if you want to add LaTeX within your code
	extendedchars=true,              % lets you use non-ASCII characters; for 8-bits encodings only, does not work with UTF-8
	frame=none,                    % adds a frame around the code
	keepspaces=true,                 % keeps spaces in text, useful for keeping indentation of code (possibly needs columns=flexible)
	keywordstyle=\color{blue},       % keyword style
	language=Prolog,                 % the language of the code
	morekeywords={},            % if you want to add more keywords to the set
	numbers=none,                    % where to put the line-numbers; possible values are (none, left, right)
	numbersep=7pt,                    % how far the line-numbers are from the code
	numberstyle=\tiny\color{mygray}, % the style that is used for the line-numbers
	rulecolor=\color{black},         % if not set, the frame-color may be changed on line-breaks within not-black text (e.g. comments (green here))
	showspaces=false,                % show spaces everywhere adding particular underscores; it overrides 'showstringspaces'
	showstringspaces=false,          % underline spaces within strings only
	showtabs=false,                  % show tabs within strings adding particular underscores
	stepnumber=1,                    % the step between two line-numbers. If it's 1, each line will be numbered
	stringstyle=\color{black},     % string literal style
	tabsize=2,                       % sets default tabsize to 2 spaces
	title=\lstname                   % show the filename of files included with \lstinputlisting; also try caption instead of title
}

\lstdefinestyle{prologAnhang}{
	backgroundcolor=\color{white},   % choose the background color; you must add \usepackage{color} or \usepackage{xcolor}
	basicstyle=\footnotesize\ttfamily,        % the size of the fonts that are used for the code
	breakatwhitespace=false,         % sets if automatic breaks should only happen at whitespace
	breaklines=true,                 % sets automatic line breaking
	captionpos=b, 					% sets the caption-position to bottom                 
	commentstyle=\color{mygreen},    % comment style
	deletekeywords={},            % if you want to delete keywords from the given language
	escapeinside={\%*}{*)},          % if you want to add LaTeX within your code
	extendedchars=true,              % lets you use non-ASCII characters; for 8-bits encodings only, does not work with UTF-8
	frame=l,                    % adds a frame around the code
	keepspaces=true,                 % keeps spaces in text, useful for keeping indentation of code (possibly needs columns=flexible)
	keywordstyle=\color{blue},       % keyword style
	language=Prolog,                 % the language of the code
	morekeywords={:-, karte, nachbar, inc, nb_setval, nb_getval, writef, loesung, member, bedroht_nicht, muster, dame},            % if you want to add more keywords to the set
	numbers=left,                    % where to put the line-numbers; possible values are (none, left, right)
	numbersep=7pt,                    % how far the line-numbers are from the code
	numberstyle=\tiny\color{mygray}, % the style that is used for the line-numbers
	rulecolor=\color{black},         % if not set, the frame-color may be changed on line-breaks within not-black text (e.g. comments (green here))
	showspaces=false,                % show spaces everywhere adding particular underscores; it overrides 'showstringspaces'
	showstringspaces=false,          % underline spaces within strings only
	showtabs=false,                  % show tabs within strings adding particular underscores
	stepnumber=1,                    % the step between two line-numbers. If it's 1, each line will be numbered
	stringstyle=\color{mymauve},     % string literal style
	tabsize=2,                       % sets default tabsize to 2 spaces
	title=\lstname                   % show the filename of files included with \lstinputlisting; also try caption instead of title
}

\lstdefinestyle{irondetectPolicy}{
	backgroundcolor=\color{myHellGelb},   % choose the background color; you must add \usepackage{color} or \usepackage{xcolor}
	basicstyle=\footnotesize\ttfamily,        % the size of the fonts that are used for the code
	breakatwhitespace=false,         % sets if automatic breaks should only happen at whitespace
	breaklines=true,                 % sets automatic line breaking
	captionpos=b, 					% sets the caption-position to bottom                 
	commentstyle=\color{mygreen},    % comment style
	deletekeywords={},            % if you want to delete keywords from the given language
	escapeinside={\%*}{*)},          % if you want to add LaTeX within your code
	extendedchars=true,              % lets you use non-ASCII characters; for 8-bits encodings only, does not work with UTF-8
	frame=single,                    % adds a frame around the code
	keepspaces=true,                 % keeps spaces in text, useful for keeping indentation of code (possibly needs columns=flexible)
	keywordstyle=\color{blue},       % keyword style
%	language=XSLT,                 % the language of the code
	morekeywords={context, hint, anomaly, signature, condition, action, rule, if, do},            % if you want to add more keywords to the set
	numbers=left,                    % where to put the line-numbers; possible values are (none, left, right)
	numbersep=7pt,                    % how far the line-numbers are from the code
	numberstyle=\tiny\color{mygray}, % the style that is used for the line-numbers
	rulecolor=\color{black},         % if not set, the frame-color may be changed on line-breaks within not-black text (e.g. comments (green here))
	showspaces=false,                % show spaces everywhere adding particular underscores; it overrides 'showstringspaces'
	showstringspaces=false,          % underline spaces within strings only
	showtabs=false,                  % show tabs within strings adding particular underscores
	stepnumber=1,                    % the step between two line-numbers. If it's 1, each line will be numbered
	stringstyle=\color{mymauve},     % string literal style
	identifierstyle=\color{black},
	tabsize=2,                       % sets default tabsize to 2 spaces
%	title=\lstname                   % show the filename of files included with \lstinputlisting; also try caption instead of title
}

\lstset{style=prologInText}

\usepackage{hyperref}
\makeindex

\renewcommand*{\mkbibnamelast}[1]{\textsc{#1}}

\newtheorem{defi}{Definition}
\newtheorem{defiTest}{Beispiel Definition}
\newtheorem{satz}[defi]{Satz}
\newtheorem{bsp}[defi]{Beispiel}
\newtheorem{AnforderungFunktional}{FA}
\newtheorem{AnforderungNichtFunktional}{NFA}
\newtheorem{AnforderungTechnische}{TA}

\newtheorem{beweis}{Beweis}

\begin{document}

	\section{Einleitung}
	Die algorithmische Zahlentheorie bildet die Grundlage der heutigen asymmetrischen Kryptographie und somit auch für einen Großteil des sicheren Datenverkehrs in Netzwerken. Ob Onlinebanking, digitale Signaturen oder virtuelle private Netzwerke, überall finden asymmetrische Verschlüsselungsalgorithmen Anwendung. Obwohl bereits Euklid ca. 300 v. Chr. zahlentheoretische Algorithmen entwickelt hat, konnte erst mit der asymmetrischen Verschlüsselung eine praktische Anwendung der Zahlentheorie gefunden werden. Zu den bedeutendsten Mathematikern, die sich mit der Zahlentheorie beschäftigten, gehören Euklid, Eratosthenes von Kyrene, Sun-Tse, Pierre de Fermat, Leonhard Euler, Carl Friedrich Gauß und David Hilbert. Trotz der langen Historie sind einige Fragen der Zahlentheorie wie z.B. die Unendlichkeit der Primzahlzwillinge oder die Goldbachsche Vermutung seit Jahrhunderten ungelöst. Erst 2002 konnten die drei indischen Wissenschaftler Manindra Agrawal, Neeraj Kayal und Nitin Saxena einen Beweis für die Existenz eines deterministischen Primzahltests liefern.\cite{Primes:is:in:P}
	
	Im folgenden werden die Grundlagen der Zahlentheorie eingeführt, um anschließend ausgewählte Algorithmen, die in der asymmetrischen Kryptographie eingesetzt werden, betrachten zu können. Zum Nachschlagen weiterer Informationen über Persönlichkeiten der Zahlentheorie mit historischer Einordnung bietet das Buch \cite{Elementare:Zahlentheorie} von Jochen Ziegenbalg einen guten Überblick.
	\section{Grundlagen}
	%TODO Grundlagen
	Grundlagen ... [TODO]
	
	\subsection{Algebraische Strukturen}
		In diesem Kapitel werden die algebraischen Strukturen: Halbgruppen, Gruppen, Ringe und Körper vorgestellt. Diese werden für ein späteres Kapitel benötigt. Die algebraischen Strukturen beschreiben ein abstraktes Rechnen mit Zahlen. Dies ermöglicht gezielter nur die Rechenregeln an sich zu untersuchen, unabhängig von der Rechengröße und der jeweiligen Operation. Ein Anwendungsbereich ist u. a. in der Kryptographie zu finden. 
	
		\subsection{Halbgruppen}
			%TODO siehe Text
			Eine Halbgruppe ist eine Menge M mit einer assoziativen Operation °. Assoziativgesetz: (a ° b) ° c = a ° (b ° c) für alle a, b, c € M. Das Zeichen ° ist Platzhalter für eine beliebige Operation. Der Wertebereich von ° ist eine Teilmenge von M so dass, a ° b € M für alle a, b €. Für das Zeichen ° werden auch die folgenden Operationszeichen verwendet: *, [MAL], +. Auch muss die Menge nicht zwangsläufig M sein. \textbf{[TODO Definition (Halbgruppe, Assoziativgesetz) S. 44]} Zum besseren verständnis konkrete Beispiele von Halbgruppen:
			
			\begin{itemize}
				\item N, Z, Q, R, C \textbf{[TODO noch die richtigen Symbole rein]} sind Halbgruppen mit der Addition als Operation, ebenso mit der Multiplikation.
				\item Wenn a ° b = |b - a| für alle a, b € Z, dann ist (Z, °) keine Halbgruppe. Da in diesem Fall (1 ° 2) ° 3 = 1 ° (2 ° 3) = 2 ist, aber 1 ° (2 ° 3) = 1 ° 1 = 0 ist.
			\end{itemize}
		
		\subsection{Monoide}
		
		
		\subsection{Gruppe}
		
		
		\subsection{Ringe}
		
		
		\subsection{Körper}
		\section*{Primzahlen}\label{Kapitel Primzahlen}
		Natürliche Primzahlen werden definiert durch Zahlen $> 1$ die nur durch Eins oder sich selbst teilbar sind. Alle natürlichen Zahlen können mit einer Multiplikation von Primzahlen erzeugt werden. Sie bilden sozusagen die Bausteine aller natürlichen Zahlen. Die Unberechenbarkeit mit der sie auftreten gibt Mathematikern schon seit Jahrtausenden Rätsel auf und ist ein Grundstein unserer heutigen Verschlüsselungsverfahren.
		In anderen Zahlensystemen als den natürlichen Zahlen ist die gewohnte Definition von Primzahlen nicht vollständig/korrekt. Sie sagt nur etwas über die Irreduzibelität eines Elements in einem Integritätsbereich aus. Da in den natürlichen Zahlen aber jedes irreduzibles Element auch prim ist, reicht diese Definition für \myMenge{N} aus.
		Für alle Integritätsbereiche gilt für die Primheit folgende Definition nach \cite{Algorithmische:Zahlentheorie}:
			
		Ein Element \myMathRM{p \in R\setminus(R^* \cup~\{0\})} heißt prim oder Primelement, wenn für alle \myMathRM{a, b \in R\setminus\{0\}} gilt:	
		\begin{displaymath}
			p~|~ab \Longrightarrow p~|~a~oder~p~|~b.
		\end{displaymath}
			
		\wup Mit Hilfe der Primzahlen kann der Restklassenring \myMenge{Z}/m\myMenge{Z} spezialisiert werden. Ist m eine Primzahl p, so ist \myMenge{Z}/p\myMenge{Z} ein Körper und wird auch \myMenge{F}\myTiefstellen{p} bzw. \myMenge{GF}\myTiefstellen{p} bezeichnet.
			
		\wup Neben den normalen Primzahlen gibt es weitere sogenannte Pseudoprimzahlen. Diese Primzahlen verhalten sich bezogen auf einen Algorithmus genauso wie echte Primzahlen, sie sind jedoch zusammengesetzt. Ein Beispiel für solche Zahlen sind Carmichaelzahlen die im Vortrag weiter thematisiert werden.
			
		\wup Leider gibt es keine bekannten effizienten Verfahren um Primzahlen zu generieren. Dennoch kann man Primzahlen recht einfach raten. Alle geratenen Zahlen müssen jedoch einem Primzahltest zur Verifizierung unterzogen werden. Im Vortrag werden ausgewählte Primzahltests zur Verifizierung vorgestellt.
		
	\section*{Diskreter Algorithmus}
		\wup In diesem Kapitel soll der diskrete Logarithmus betrachtet werden. Es werden Anwendungsbeispiele aufgezeigt und genau erläutert, weswegen sich der diskrete Logarithmus besonders gut in der Kryptografie als Einwegfunktion eignet. Zu Beginn wird das grundsätzliche Problem beim diskreten Logarithmus an einem Beispiel praxisnahe erläutert, um im Anschluss auf algorithmische Lösungen einzugehen wie der diskreten Logarithmus in \myZPStern~und auf elliptische Kurven berechnet werden kann. Hierfür wird der Baby-Step-Giant-Step-Algorithmus vorgestellt.
	
	\section*{Das Problem des diskreten Logarithmus im Detail}\label{Das Problem des diskreten Logarithmus im Detail}
		Es existiert eine Primzahl p, ein erzeugendes Element g für \myZPStern~sowie eine ganze Zahl x. Zu der diskreten Exponentialfunktion $g^x~mod~p$ gibt es die diskrete Logarithmusfunktion, die zu einem gegebenen y und g, x beschreibt. Somit ist x der diskrete Logarithmus von y zur Basis g, ($y = g^x~mod~p$). Jede Zahl aus \myZPStern~lässt sich als Potenz von g darstellen, wenn g ein erzeugendes Element von \myZPStern~ist. Ist dies nicht der Fall, so muss es nicht zu jedem $y$ \myin \myZPStern~einen diskreten Logarithmus geben.~\cite{Kryptografie:in:Theorie:und:Praxis} Um das eigentliche Problem des diskreten Logarithmus zu verstehen ist es hilfreich, die Logarithmen in \myMenge{R} mit Logarithmen in \myZPStern~gegenüberzustellen.
		
		\begin{minipage}{0.44\textwidth}
			\begin{equation}
			\begin{aligned}
			y &= g^x\\
			1024 &= 2^x\\
			x &= log_2 1024\\
			x &= 10
			\end{aligned}
			\label{Gleichung Log in Z}
			\end{equation}
		\end{minipage}
		\begin{minipage}{0.44\textwidth}
			\begin{equation}
			\begin{aligned}
			y &= g^x~mod~p\\
			10 &= 2^x~mod~11\\
			x &= log_2 10\\
			x &= 3.32193...~\lightning
			\end{aligned}
			\label{Gleichung Log in ZP}
			\end{equation}
		\end{minipage}
		
		\wupp Die Gleichungen aus \myRefGleichung{Gleichung Log in Z} zeigen wie der Logarithmus von x zur Basis g, mit der Logarithmusfunktion berechnet werden kann. Auf diese Weise sind Gleichungen für die positiven reellen Zahlen \myMenge{R^+} immer eindeutig lösbar. Für die Gleichungen aus \myRefGleichung{Gleichung Log in ZP} wird ebenfalls mit der Logarithmusfunktion versucht, den Logarithmus von 10 zur Basis 2 zu erhalten. Als Ergebnis erhält man eine Zahl die nicht in \myZPStern~enthalten ist, was auch nicht verwundert, da die Logarithmusfunktion $mod~11$ gar nicht berücksichtigt. Genau hier ist das grundsätzliche Problem beim diskreten Logarithmus. Es gibt keine mathematische Rechenoperation die, es ermöglicht, den diskreten Logarithmus in einem endlichen Körper mit nur einem Rechenschritt zu berechnen. Um dennoch eine Lösung zu erhalten, scheint das Enumerationsverfahren das naheliegendste zu sein. Hierbei werden einfach alle Werte, die für $x$ in Frage kommen, durchprobiert. So ist die Lösung der Gleichungen aus \myRefGleichung{Gleichung Log in ZP}, $x = 5$.~\cite{DLP:ECDLP:Probleme:und:Loesungen} Für sehr große Gruppen, wo x beispielsweise eine 160 Bit große Zahl ist, gibt es bis heute keine Algorithmen die den diskreten Logarithmus effizient berechnen.\cite{Kryptografie:in:Theorie:und:Praxis} Es gibt allerdings eine ganze Reihe von Algorithmen, die in der Lage sind, den diskreten Logarithmus gezielter zu berechnen als das naive Ausprobieren, siehe hierfür die Präsentation.
		
		%TODO nchmal schauen das will ich nochmal anders haben schauen ob die Reihenfolge geändert wird für (Das Problem des diskreten Logarithmus im Detail) <=> (Diskreter Logarithmus auf elliptische Kurven)
		
		\subsection*{Diskreter Logarithmus auf elliptische Kurven}
			In diesem Unterkapitel soll kurz beschrieben werden, was genau der Logarithmus auf einer elliptischen Kurve über \myZPStern~ist. Für eine elliptische Kurve $E$ über den Primkörper \myZPStern~mit den Punkten $P, Q \in E$(\myZPStern), gibt es ein $k \in$ \myZPStern~sodass die folgende Gleichung erfüllt ist:
			\begin{displaymath}
				Q = kP
			\end{displaymath}
			So wird die Zahl $k$ als diskreter Logarithmus von $Q$ zur Basis $P$ bezeichnet. Den diskreten Logarithmus aus $E$(\myZPStern) zu bestimmen, ist noch einmal ungleich komplexer als aus \myZPStern. Dieses Berechnungsproblem wird \textbf{E}lliptic \textbf{C}urve \textbf{D}iscrete \textbf{L}ogarithm \textbf{P}roblem, kurz \textbf{ECDLP}, genannt.
	\section*{Ausblick}~\label{Kapitel Fazit}
	Mit den hier erlangten Grundlagen werden während des Vortrags fortgeschrittene Sätze und Definitionen gegeben, um die Eingangs erwähnten Primzahltest, Logarithmen und Schlüsselaustauschverfahren zu analysieren.

	Abschließend werden Laufzeiten der Algorithmen betrachtet und ein Fazit für die Sicherheit von asymmetrischen Kryptographieverfahren gegeben.
%	\section*{Anhang}\label{Kapitel Anhang}
	TODO Anhang...
	
	\section*{Literaturverzeichnis}
	\printbibliography[type=book,heading=subbibliography,title={Bücher}]
	\printbibliography[type=thesis, heading=subbibliography,title={Promotionsarbeiten}]
	\printbibliography[type=mvbook, heading=subbibliography,title={Abschlussarbeiten}]
	\printbibliography[type=booklet, heading=subbibliography,title={Berichte}]
	\printbibliography[type=unpublished, heading=subbibliography,title={Unveröffentlichtes}]
	\printbibliography[type=manual,heading=subbibliography,title={Technische Dokumentationen}]
	\printbibliography[type=article,heading=subbibliography,title={Zeitschriftenaufsätze}]
	\printbibliography[type=online,heading=subbibliography,title={Websiten}]
	\printbibliography[type=misc,heading=subbibliography,title={Sonstige}]
	\printbibliography[nottype=book, nottype=thesis, nottype=mvbook, nottype=booklet, nottype=unpublished, nottype=manual, nottype=article, nottype=online, nottype=misc, heading=subbibliography, title={Skripte}]

\end{document}