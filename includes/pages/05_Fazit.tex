\section{Fazit}
	Diese Ausarbeitung hat in Kapitel \ref{Kapitel Grundlagen} Grundlagen Wissen vermittelt, über Algebraische Strukturen, Euklidischen Algorithmus sowie zur Euler’schen \myPhi -Funktion. Weiter wurde auch ein kleiner Teilbereich aus den elliptischen Kurven vorgestellt um die nötigen Grundlagen zu schaffen für das ECC. In Kapitel \ref{Kapitel Primzahlen} wurden die Primzahlen genauer analysiert. Dabei ist die Primfaktorzerlegung essenziell für die Zahlentheorie und ermöglichen erst kryptographische Algorithmen die auf eine ineffiziente Berechnung der Primfaktoren aufbauen. Das grundlegende Problem vom diskreten Logarithmus wurde in Kapitel \ref{Kapitel Diskreter Logarithmus} ausführlich erläutert. Des weiteren wurden Algorithmen vorgestellt mit dessen Hilfe es möglich ist den diskreten Logarithmus zu berechnen. Besonders der Index-Calculus-Algorithmus ist den allgemein verwendbaren Algorithmen deutlich überlegen, den dieser hat eine Laufzeit von etwas $O(exp(\sqrt[]{2 \cdot ln~p \cdot ln~ln~p}))$. Eine genaue Beschreibung diese Algorithmus ist dem Buch \cite{Einfuehrung:in:die:Kryptographie} zu entnehmen. Dieser ist auch dafür mitverantwortlich weswegen elliptische Kurven gegenüber den multiplikativen Gruppen der endlichem Körper, für kryptografische Anwendungsfälle \myAnfuehrungszeichen{sicherer} sind. Der Baby-Step-Giant-Step-Algorithmus kann sowohl für das DLP sowie für das ECDLP eingesetzt werden, besitzt allerdings nur eine Laufzeit von $O(\sqrt[]{p-1})$. Der Index-Calculus-Algorithmus hingegen lässt sich nicht auf elliptischen Kurven umschreiben, hier bleiben nur die wesentlich langsameren allgemeinen Verfahren.\cite{Mathematisch:fuer:fortgeschrittene:Anfaenger}
	
	Verschlüsselung von Daten ist aus der heutigen vernetzten Welt nicht mehr wegzudenken. Aus diesem Grund sind heutige Verschlüsselungstechniken die auf faktorisierung von Primzahlen oder dem diskreten Logarithmus Problem beruhen zum Standard geworden. Dennoch hat die Verschlüsselung noch lange nicht ihren Höhepunkt erreicht, wie das ECDLP zeigt. Noch schnellere und effizientere und somit \myAnfuehrungszeichen{sicherer} Verschlüsselungstechniken werden benötigt um auch zukünftigen Anforderungen gerecht zu werden.