\section{Fazit}
	Fazit zum DLP:
	
	Es gibt noch eine reihe weiterer Algorithmen zum lösen des DLP. Besonders der Index-Calculus-Algorithmus ist den allgemein verwendbaren Algorithmen deutlich überlegen, den dieser hat eine Laufzeit von etwas $O(exp(\sqrt[]{2 \cdot ln~p \cdot ln~ln~p}))$. Eine genaue Beschreibung diese Algorithmus ist dem Buch \cite{Einfuehrung:in:die:Kryptographie} zu entnehmen. Dieser ist auch dafür mitverantwortlich weswegen elliptische Kurven gegenüber den multiplikativen Gruppen der endlichem Körper, für kryptografische Anwendungsfälle \myAnfuehrungszeichen{sicherer} sind. Der Baby-Step-Giant-Step-Algorithmus kann sowohl für das DLP sowie für das ECDLP eingesetzt werden, besitzt allerdings nur eine Laufzeit von $O(\sqrt[]{p-1})$. Der Index-Calculus-Algorithmus hingegen lässt sich nicht auf elliptischen Kurven umschreiben, hier bleiben nur die wesentlich langsameren allgemeinen Verfahren.\cite{Mathematisch:fuer:fortgeschrittene:Anfaenger}