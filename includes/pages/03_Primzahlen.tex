\section{Primzahlen}
	Primzahlen werden definiert durch Zahlen die nur durch Eins oder sich selbst teilbar sind.
	
	Des weiteren gilt: Wenn m des Restklassenring \myMenge{Z}/m\myMenge{Z} eine Primzahl ist, ist \myMenge{Z}/m\myMenge{Z} ein Körper und wird auch \myMenge{F}\myTiefstellen{p} bezeichnet.

	\subsection{Primfaktorzerlegung}
	\subsection{Satz von Euler und Kleiner Satz von Fermat}
	\subsection{Carmichaelzahl}
	\subsection{Primzahltests}
		\subsubsection{Sieb des Eratosthenes}
		\subsubsection{Miller-Rabin-Test}
		\subsubsection{AKS-Test}