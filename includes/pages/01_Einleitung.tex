\section{Einleitung}
	Die algorithmische Zahlentheorie bildet die Grundlage der heutigen asymmetrischen Kryptographie und somit auch für einen Großteil des sicheren Datenverkehrs in Netzwerken. Ob Onlinebanking, digitale Signaturen oder virtuelle private Netzwerke, überall finden asymmetrische Verschlüsselungsalgorithmen Anwendung. Obwohl bereits Euklid ca. 300 v. Chr. zahlentheoretische Algorithmen hat, konnte erst mit der asymmetrischen Verschlüsselung eine praktische Anwendung der Zahlentheorie gefunden werden. Zu den bedeutendsten Mathematikern die sich mit der Zahlentheorie beschäftigten gehören Euklid, Eratosthenes von Kyrene, Diophantos von Alexandria, Sun-Tse, Pierre de Fermat, Leonhard Euler, Carl Friedrich Gauß und David Hilbert. Trotz der langen Historie sind einige Fragen der Zahlentheorie wie z.B. die Unendlichkeit der Primzahlzwillinge oder die Goldbachsche Vermutung seit Jahrhunderten ungelöst. Erst 2002	konnten die drei indischen Wissenschaftlern Manindra Agrawal, Neeraj Kayal und Nitin Saxena einen Beweis für die Existenz eines deterministischen Primzahltests liefern.
	
	Im folgenden werden die Grundlagen der Zahlentheorie eingeführt, um anschließend ausgewählte Algorithmen, die in der asymmetrischen Kryptographie eingesetzt werden, betrachten zu können. Für Information weiterer Persönlichkeiten der Zahlentheorie mit historischer Einordnung, bietet das Buch[Elementare Zahlentheorie] von Jochen Ziegenbalg einen guten Überblick.
	
	 