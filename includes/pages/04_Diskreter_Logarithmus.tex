\section{Diskreter Logarithmus}
	%TODO Diskreter Logarithmus
	Diskreter Logarithmus ... [TODO]
	
	\subsection{Elliptischen Kurven Grundlagen}
		In diesem Kapitel sollen nur die Grundlagen von elliptischen Kurven näher gebracht werden, um so die Elliptic Curve Cryptography, kurz ECC, verstehen zu können. Der Vorteil beim ECC-Verfahren im Vergleich zum RSA-Verfahren, liegt darin das die Schlüssellänge deutlich kürze ausfallen kann ohne dabei an Sicherheit zu verlieren. Ein RSA-Schlüssel mit 1024 Bit ist etwa so sicher wie ein Schlüssel aus einer elliptischen Kurve mit gerade mal ca. 160 Bit. Dazu kommt das der Rechenaufwand und Speicherbedarf beim ECC-Verfahren wesentlich geringer ist als beim RSA-Verfahren. So kann ECC in Smartcards und Mobiltelefonen genutzt werden.\cite{Information:und:Kommunikation}
				
		Um die Funktionsweise der elliptischen Kurven in ihrer vollen Breite und Tiefe zu verstehen ist dafür eine sehr komplexe Mathematik notwendig. Innerhalb dieser Seminararbeit kann dieses Thema nicht Breiter und Tiefer durchleuchtet werden und es sei ihr auf die Folgende Literatur verwiesen: \textbf{[TODO Quellen]}
		
		Eine Elliptische Kurve ist eine ebene Kurve und durch nachfolgender Gleichung beschrieben: $y^2 = x^3 + ax +b$. Damit ist eine Menge Z aller Punkte P(x, y) die auf der elliptischen Kurve liegen definiert. Wichtig dabei ist das die Kurvenparameter a und b so gewählt sind das die partiellen Ableitungen nach x und nach y auf keinem Punkt der Kurve gleichzeitig null sind, dazu später mehr.
		Das Addieren von zwei Punkten, die auf der elliptischen Kurve liegen, ergibt wieder einen Punkt welcher ebenfalls auf der Kurve liegt.\cite{Information:und:Kommunikation} Mit Addition ist das Verknüpfen von zwei Punkten gemeint, man könnte es auch als Multiplikation bezeichnen, im weiteren verlauf wird es als Addieren bezeichnet. In beiden fällen hat es nichts mit den bekannten Operationen auf Zahlen zu tun. Das Addieren von zwei Punkten ist vielmehr geometrisch definiert:
		
		[TODO Bild von S. 252 \cite{Information:und:Kommunikation}]
		
		\begin{quote}
			\begin{defi}
				Durch die gegebenen Punkte P und Q wird eine Gerade gelegt, welche die Kurve in einem dritten Punkt R schneidet. Dieser wird anschließend an der x-Achse 
				gespiegelt. Als Ergebnis erhält man den Punkt S, welcher als Addition von P und Q bezeichnet wird.\cite{Information:und:Kommunikation}
			\end{defi}
		\end{quote}
		
		Die so definierte Addition ist kommutativ, zur Erinnerung: P + Q = Q + P. Nicht für alle elliptischen Kurven kann eine Addition von Punkten durchgeführt werden. Wie oben bereits erwähnt dürfen die partiellen Ableitungen nach x und nach y auf keinem Punkt der Kurve gleichzeitig null sein. Anders ausgedrückt die Kurve darf sich nicht selbst schneiden, ansonsten kann die Additionsoperation nicht für beliebige Punkte durchgeführt werden.
		Zusätzlich muss beachtet werden, dass bei einer Addition von zwei Punkten die nachfolgenden Spezialfälle auftreten können\cite{Information:und:Kommunikation}:
		
		\begin{itemize}
			\item Wenn für die beiden zu Addierenden Punkten Q = P gilt, wird die Tangente an der Kurve im Punkt P verwendet. Dabei entsteht der der Schnittpunkt mit der Kurve in R und durch Spiegelung resultiert daraus S = P + P = 2P.
			\item Sollten die x-Koordinaten beider zu addierender Punkte gleich sein, so dass (XQ = XP) gilt, entsteht eine vertikale Gerade und die Kurve wird kein weiteres mal geschnitten. Für diesen Fall wird die elliptische Kurve um einen weiteren Punkt O\textbf{[TODO anderes Symbol]}, welcher im Unendlichen liegt, ergänzt. Die Addition von Punk P mit O\textbf{[TODO anderes Symbol]} ist so definiert das man wiederum P als Ergebnis erhält (P + O\textbf{[TODO anderes Symbol]} = P). Somit ist O\textbf{[TODO anderes Symbol]} das neutral Element der Addition. Es gilt also: P + Q = O\textbf{[TODO anderes Symbol]} wenn die x-Koordinaten von P und Q gleich sind. Daraus folgt das Q das inverse Element vo P ist und es gilt: Q = -P.
		\end{itemize}
		
		Das Addieren eines Punktes P mit einem Skalar k \myin \{1, 2, 3 ...\} wird als wiederholte Addition definiert. Beispiel: kP = P1 + P2 + ... + Pk.
		
	\subsection{Asymmetrische Verschlüsselung mit Elliptischen Kurven}
		Um Elliptischen Kurven für Asymmetrische Verschlüsselung einsetzen zu können muss in einem endlichen Körper gerechnet werden um Rundungsfehler zu vermeiden. Bei der Addition und Multiplikation in endlichen Körpern sind diese so definiert, dass das Ergebnis immer wider ein Element des endlichem Körpers ist. Bei der Addition oder Multiplikation von zwei Elementen in einem endlichen Körper muss eine weitere Operation durchgeführt werden: \textit{mod |Z|}. Der resultierende Rest ist in jedem Fall wieder ein Element aus Z. Für die Addition besitzt jedes Element ein inverses Element -a, damit gilt für die Subtraktion: b - a = b + (-a). Bei der Multiplikation ist das inverse Element $a^-1$, damit gilt für die Division: b / a = b \mycdot $a^1$. Siehe auch konkretes Beispiel in \cite{Information:und:Kommunikation} S. 154 - 257.
		
		Um elliptische Kurven für kryptologische Anwendungsfälle zu nutzen, muss die Ordnung eines Punktes der elliptischen Kurve berechnet werden.
		
		\begin{quote}
			\begin{defi}
				Die Ordnung eines Punktes ist die Anzahl der Punkte, die durch fortwährender Addition dieses Punktes, erzeugt werden.\cite{Information:und:Kommunikation}
			\end{defi}
		\end{quote}
		
		Dazu folgendes Beispiel: P + P = 2P -> 2P +P = 3P -> ... -> xP + P = (x+1)P. P ist dabei immer ein Punkt auf der elliptischen Kurve. Irgendwann ist xiP = O\textbf{[TODO anderes Symbol]} und damit hat der Punkt P die Ordnung xi.
		
		\subsubsection{Schlüsselaustausch mit elliptischen Kurven}
			Zuerst muss der Körper bestimmt werden und eine elliptische Kurve, dazu wählt man ein große Primzahl und die Kurvenparameter a und b. Weiter wird nun ein Erzeugerpunkt G vereinbart, dabei soll die Ordnung des Punktes G möglichst groß und eine Primzahl sein. A wählt eine geheime ganze Zahl nA welche kleiner sein muss als n und berechnet daraus den öffentlichen Schlüssel PA = nA \mycdot G. Teilnehmer B mach das gleich jeweils mit nB und PB. PA und PB können nun über eine unsichere Leitung ausgeschaut werden. Nun kann Teilnehmer A den Schlüssel K = nA \mycdot PB berechnen. B berechnet ebenfalls K mit nB und PA. So habe dabei ein und das selbe geheime K berechnet.
			
			Es folgt der Beweis das A und B wirklich das gleiche K berechnet haben müssen:
			\begin{quote}
				\begin{beweis}
					A berechnet K = nA \mycdot PB. Für PB wurde ursprünglich von Teilnehmer B berechnet mit nB \mycdot G. Die kann in die berechnung für K in PB eingesetzt werden so das daraus folgt: K = nA \mycdot PB = nA \mycdot (nB \mycdot G). Das gleiche Prinzip angewendet für die Berechnung von K von Teilnehmer B ergibt: K = nB \mycdot PA = nB \mycdot (nA \mycdot G). Unter Berücksichtigung des Assoziativgesetzes können diese beiden Gleichungen, gleich gesetzt werden: nA \mycdot (nB \mycdot G) = nB \mycdot (nA \mycdot G).
				\end{beweis}
			\end{quote}
			
			Asymmetrische Verschlüsselungsverfahren basieren auf Einwegfunktionen. Wobei es nicht allzu schwierig ist k \mycdot P zu Berechnen allerdings ist das Berechnen von k aus k \mycdot P und P sehr aufwendig. Anzumerken ist das diese aussage allerdings bis heute noch nicht bewiesen wurde.