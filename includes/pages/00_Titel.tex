%TODO ...Problemlösenden Algorithmen für ...
\title{Algorithmische Zahlentheorie\titlenote{Genau Betrachtungen der Problemlösenden Algorithmen für Effiziente Primzahltests(Miller-Rabin-Test / AKS-Test) sowie den diskreten Logarithmus(Baby-Step-Giant-Step-Algorithmus / Index-Calculus-Algorithmus) sowie die dafür nötigen Grundlagen}}
\subtitle{Seminararbeit im Masterstudiengang Angewandte Informatik \\WS 2015/16 Hochschule Hannover
	\titlenote{ATM ka was man hier noch beschreiben könnte erstmal [TODO]}}

%
% You need the command \numberofauthors to handle the 'placement
% and alignment' of the authors beneath the title.
%
% For aesthetic reasons, we recommend 'three authors at a time'
% i.e. three 'name/affiliation blocks' be placed beneath the title.
%
% NOTE: You are NOT restricted in how many 'rows' of
% "name/affiliations" may appear. We just ask that you restrict
% the number of 'columns' to three.
%
% Because of the available 'opening page real-estate'
% we ask you to refrain from putting more than six authors
% (two rows with three columns) beneath the article title.
% More than six makes the first-page appear very cluttered indeed.
%
% Use the \alignauthor commands to handle the names
% and affiliations for an 'aesthetic maximum' of six authors.
% Add names, affiliations, addresses for
% the seventh etc. author(s) as the argument for the
% \additionalauthors command.
% These 'additional authors' will be output/set for you
% without further effort on your part as the last section in
% the body of your article BEFORE References or any Appendices.

\numberofauthors{2} %  in this sample file, there are a *total*
% of EIGHT authors. SIX appear on the 'first-page' (for formatting
% reasons) and the remaining two appear in the \additionalauthors section.
%
\author{
	% You can go ahead and credit any number of authors here,
	% e.g. one 'row of three' or two rows (consisting of one row of three
	% and a second row of one, two or three).
	%
	% The command \alignauthor (no curly braces needed) should
	% precede each author name, affiliation/snail-mail address and
	% e-mail address. Additionally, tag each line of
	% affiliation/address with \affaddr, and tag the
	% e-mail address with \email.
	%
	% 1st. author
	\alignauthor
	Marius Rohde\titlenote{Marius Rohde ... [TODO]}\\
	\affaddr{Hochschule Hannover}\\
	\affaddr{Fakultät IV - Wirtschaft und Informatik}\\
	\affaddr{30459 Hannover}\\
	\email{Marius.Rohde@stud.HS-Hannover.de}
	% 2nd. author
	\alignauthor
	Marcel Reichenbach\titlenote{Marcel Reichenbach wünscht allen Lesern und Leserinnen ein Frohes Weihnachten und einen guten Rutsch ins neue Jahr und sowie im nächsten Jahr bald Master zu sein}\\
	\affaddr{Hochschule Hannover}\\
	\affaddr{Fakultät IV - Wirtschaft und Informatik}\\
	\affaddr{30459 Hannover}\\
	\email{Marcel.Reichenbach@stud.HS-Hannover.de}
}


\maketitle