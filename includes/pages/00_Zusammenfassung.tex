\begin{abstract}
	Diese Arbeit beschäftigt sich mit der theoretischen Grundlage für die asymmetrische Kryptografie. Es werden zunächst die benötigten Grundlagen der Zahlentheorie gegeben, um anschließend einen genaueren Blick auf die Algorithmen zur Primzahlerkennung, des Diffie-Hellmann Austausches und der Anwendung elliptischer Kurven zu geben. Neben der Ausarbeitung verwendeter Algorithmen wird auch die Laufzeit dieser untersucht. Zufallszahlengeneratoren werden nicht betrachtet.
\end{abstract}