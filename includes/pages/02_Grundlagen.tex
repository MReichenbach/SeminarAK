\section{Grundlagen}
	%TODO Grundlagen
	Grundlagen ... [TODO]
	
	\subsection{Algebraische Strukturen}
		In diesem Kapitel werden die algebraischen Strukturen: Halbgruppen, Gruppen, Ringe und Körper vorgestellt. Diese werden für ein späteres Kapitel benötigt. Die algebraischen Strukturen beschreiben ein abstraktes Rechnen mit Zahlen. Dies ermöglicht gezielter nur die Rechenregeln an sich zu untersuchen, unabhängig von der Rechengröße und der jeweiligen Operation. Ein Anwendungsbereich ist u. a. in der Kryptographie zu finden. 
	
		\subsection{Halbgruppen}
			%TODO siehe Text
			Eine Halbgruppe ist eine Menge M mit einer assoziativen Operation °. Assoziativgesetz: (a ° b) ° c = a ° (b ° c) für alle a, b, c € M. Das Zeichen ° ist Platzhalter für eine beliebige Operation. Der Wertebereich von ° ist eine Teilmenge von M so dass, a ° b € M für alle a, b €. Für das Zeichen ° werden auch die folgenden Operationszeichen verwendet: *, [MAL], +. Auch muss die Menge nicht zwangsläufig M sein. \textbf{[TODO Definition (Halbgruppe, Assoziativgesetz) S. 44]} Zum besseren verständnis konkrete Beispiele von Halbgruppen:
			
			\begin{itemize}
				\item N, Z, Q, R, C \textbf{[TODO noch die richtigen Symbole rein]} sind Halbgruppen mit der Addition als Operation, ebenso mit der Multiplikation.
				\item Wenn a ° b = |b - a| für alle a, b € Z, dann ist (Z, °) keine Halbgruppe. Da in diesem Fall (1 ° 2) ° 3 = 1 ° (2 ° 3) = 2 ist, aber 1 ° (2 ° 3) = 1 ° 1 = 0 ist.
			\end{itemize}
		
		\subsection{Monoide}
		
		
		\subsection{Gruppe}
		
		
		\subsection{Ringe}
		
		
		\subsection{Körper}