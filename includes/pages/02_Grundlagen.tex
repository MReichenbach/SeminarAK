\section{Grundlagen}
	In diesem Kapitel werden die Grundlagen zur Analyse von Primzahlen und dem diskreten Logarithmus gegeben. Dazu werden im allg. nur die Definitionen und Sätze geliefert. Für Beweise der hier angeführten Sätze sei auf die Bücher [x],[y],[z] verwiesen. Es wird vorausgesetzt das die Rechenvorschrift des Modulo bekannt ist.
	
	Für dies Ausarbeitung wurden viele Quellen unterstützend verwendet. Da es Notations-Unterschiede der einzelnen Quellen gibt führt dies dazu, dass sich nicht an die Notation aller genutzten Quellen gehalten werden kann. Speziell in \cite{Erste:Hilfe:in:Linearer:Algebra} wird eine multiplikative Gruppe aller Einheiten in $G$ mit $G^X$ bezeichnet. Von dieser Notation wird abgesehen und die aus den anderen Quellen weiter verbreitete Notation $G^*$ verwendet.
	
	\subsection{Algebraische Strukturen}
		%TODO wenn nötig und möglich noch mal zu den anderen Struktruen je 1 -2 konkrete Beispiele mit aufnehmen wie bei Halbgruppen
		In diesem Kapitel werden die algebraischen Strukturen: Halbgruppen, Gruppen, Ringe und Körper vorgestellt. Diese werden für ein späteres Kapitel benötigt. Die algebraischen Strukturen beschreiben ein abstraktes Rechnen mit Zahlen. Dies ermöglicht gezielter nur die Rechenregeln an sich zu untersuchen, unabhängig von der Rechengröße und der jeweiligen Operation. Ein Anwendungsbereich ist u. a. in der Kryptographie zu finden.~\cite{Kryptographie:und:Algorithmen}
	
		\subsubsection{Halbgruppen}
			Eine Halbgruppe ist eine Menge M mit einer assoziativen Operation \mycircOhne, geschrieben mit (M, \mycircOhne) oder einfach nur M. Zur Erinnerung, das Assoziativgesetz besagt: (a \mycirc b) \mycirc c = a \mycirc (b \mycirc c) für alle a, b, c \myin M. Es gilt für sämtliche Elemente einer Halbgruppe. Das Zeichen \mycirc ist Platzhalter für eine beliebige Operation. Der Wertebereich von \mycirc ist eine Teilmenge von M so dass, a \mycirc b \myin M für alle a, b \myin M. Für das Zeichen \mycirc werden auch die folgenden Operationszeichen verwendet: $*$, \mycdotOhne, +. Auch muss die Menge nicht zwangsläufig M sein.~\cite{Erste:Hilfe:in:Linearer:Algebra}
			
			Durch das Assoziativgesetz können also Klammern weggelassen werden. Zum besseren Verständnis konkrete Beispiele von Halbgruppen~\cite{Erste:Hilfe:in:Linearer:Algebra}:
			
			\begin{itemize}
				\item \myMenge{N}, \myMenge{Z}, \myMenge{Q}, \myMenge{R}, \myMenge{C} sind Halbgruppen mit der Addition als Operation, ebenso wie mit der Multiplikation.
				\item Wenn a \mycirc b = |b - a| für alle a, b \myin \myMenge{Z}, dann ist (\myMenge{Z}, \mycircOhne) keine Halbgruppe. Da in diesem Fall (1 \mycirc 2) \mycirc 3 = 1 \mycirc 3 = 2 ist, aber 1 \mycirc (2 \mycirc 3) = 1 \mycirc 1 = 0 ist. Ein verändern der Klammerung ergibt unterschiedliche Ergebnisse, somit ist das Assoziativgesetz nicht mehr gewährleistet, wodurch \myMenge{Z} in diesem Fall keine Halbgruppe mehr sein kann.
			\end{itemize}

			Sobald es ein neutrales Element in einer Halbgruppe gibt heißt dieses \textbf{Monoid}. Ein neutrales Element ist immer dann gegeben wenn es ein e \myin M gibt, sodass für alle a \myin M gilt: a \mycirc e = e \mycirc a = a. Dieses weitere Axiom muss von jeder Halbgruppe erfüllt werden um ein Monoid zu sein. Als Zeichen für ein Monoid wird neben e oft auch 1 verwendet, bei den Operationszeichen \mycircOhne, \mycdotOhne, $*$. Wird das + als Operationszeichen verwendet ist oft 0 das neutrale Element.~\cite{Erste:Hilfe:in:Linearer:Algebra}

			Wenn ein Monoid auch das folgende Axiom erfüllt, ist es eine \textbf{Gruppe}. Existiert für alle a \myin G ein b \myin G, sodass a \mycirc b = b \mycirc a = e gilt, so heißt b invers zu a.~\cite{Erste:Hilfe:in:Linearer:Algebra}
		
		\subsubsection{Ringe}
			In einem Ring als Algebraische Struktur sind mehr als nur eine Operation vorhanden. Eine Menge R mit den zwei Operationen + und \mycdot auf R ist genau dann ein Ring wenn folgenden drei Bedingungen gelten~\cite{Erste:Hilfe:in:Linearer:Algebra}:
			
			\begin{itemize}
				\item (R, +) ist eine abelsche Gruppe. (Eine Operation \mycirc auf einer Menge M heiß kommutativ oder abelsch, wenn: a \mycirc b = b \mycirc a für alle a, b \myin M gilt.)
				\item (R, \mycdotOhne) ist ein Monoid
				\item Für alle a, b, c \myin R gilt: a(b + c) = ab + ac, (a + b)c = ac + bc (Distributivgesetze)
			\end{itemize}
			
			Zusätzlich heißt ein Ring kommutativ, wenn die Operation \mycdot kommutativ ist. Ein Ring besitzt also immer eine kommutative Addition und eine nicht notwendigerweise kommutative Multiplikation. Die beiden Distributivgesetze verbinden diese beiden Operationen miteinander. Ein Ring heißt nullteilerfrei wenn a \mycdot b = 0 ist und dadurch impliziert wird, dass a = 0 oder b = 0 sein muss, für alle a, b \myin R. Wenn es für ein a \myin R ein b gibt, so dass ab = ba = 1 gilt, dann ist a invertierbar oder eine Einheit. Alle Elemente im Monoid (R, \mycdotOhne) wo dies zutrifft sind in einer multiplikativen Gruppe zusammengefasst, bezeichnet wird diese mit R$^*$.~\cite{Erste:Hilfe:in:Linearer:Algebra}
			
			Es gibt spezielle Ringe, die sogenannten Körper. Ist eine Menge (K, +, \mycdotOhne) ein Ring und ist ($K / \{0\}$, \mycdotOhne) eine kommutative Gruppe, heißt K ein Körper. Alle von Null verschiedenen Elementen sind Einheiten und es gilt: K* = $K / \{0\}$.~\cite{Erste:Hilfe:in:Linearer:Algebra}
			
		\subsubsection{Integritätsbereiche}
			